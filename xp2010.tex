\documentclass[lnbip]{svmultln}

\usepackage{makeidx}
\usepackage{graphicx,url}

\sloppy

\begin{document}

\title{Open Source and Agile Methods:\\Two worlds closer than it
  seems}

\titlerunning{Open Source and Agile Methods}

\author{Hugo Corbucci\inst{1} and Alfredo Goldman\inst{1}}

\authorrunning{Hugo Corbucci et al.}

\tocauthor{Hugo Corbucci, Alfredo Goldman}

\institute{Instituto de Matem\'{a}tica e Estat\'{i}stica (IME)\\
  Universidade de S\~{a}o Paulo (USP) - Brazil\\
  \email{corbucci@ime.usp.br} and \email{gold@ime.usp.br} }
 
\maketitle

\begin{abstract}
  Agile methods and open source software communities have different
  approaches to produce high quality and successful software. However
  agile methodologies are not very difused in open source communities
  nor the members of those communities follow many agile
  practices.

  \keywords{agile software development, open source software,
    distributed agile, }
\end{abstract}

\section{Introduction}

Typical Open Source (OS) projects (the scope of of OS project will be
narrowed according to Section \ref{subsec:os-scope}) usually receive
the collaboration of many geographically distant people
\cite{report:dempsey1999}. At first glance, this argument could
indicate that such projects are not candidates for the use of agile
methods since some basic values seem to be missing. In this case, the
distance and diversity separating developers deteriorates
communication, a very important value within agile methods. However,
it is common to identify some principles presented by the agile
manifesto \cite{url:agilemanifesto} on many open source software
projects \cite{gabriel2005}. Being ready for changes, working with continuous feedback,
delivering real features, respecting collaborators and users and
facing challenges are expected attitudes from agile developers
naturally found in the Free and Open Source Software (FOSS)
communities.

During a workshop \cite{conference:oopsla2007} about ``No Silver
Bullets'' \cite{brooks1987} held at OOPSLA 2007, agile methods and OS
software development were mentioned as two failed silver bullets
having both brought great benefit to the software community. During
the same workshop the question was raised whether the use of several
failed silver bullets simultaneously could not raise production levels
by an order of magnitude. This is an attempt to suggest one of those
merges to partially tackle software development problems.

% Write the agenda

\section{Scopes}
\label{sec:scope}

In order to start talking about OS and agile methods, it is necessary
to first define which part of each will be analysed in this
work. Agile methods in the scope of ths work are described in Section
\ref{subsec:agile-scope} while the OS projects studied in this work
are described in Section \ref{subsec:os-scope}.

\subsection{Agile methods scope}
\label{subsec:agile-scope}

Thoughout this work, any software engineering method that follows the
principles of the agile manifesto \cite{url:agilemanifesto} will be
considered and treated as an agile method. Focus will be given on the
most known methods, such as eXtreme Programming (XP) \cite{XP2002},
Scrum \cite{schwaber2004} and the Crystal family
\cite{cockburn2002}. Closely related ideas will also be mentioned from
the wider Lean philosophy \cite{ohno1998} and its application to
software development \cite{poppendieck2005}.

\subsection{Open source scope}
\label{subsec:os-scope}

The terms ``Open source software'' and ``Free software'' will be
considered the same in this work although they have some differences
in their specific contexts \cite[Ch. 1, Free Versus Open
source]{fogel2005}. Projects will be called to be open source (or
free) if their source code is available and modifiable by anyone with
the required technical knowledge, without prior consent from the
original author and without any charge.

OS projects essentially controlled by a single company will not be
addressed in this work. The reason for such reduction of scope is that
projects controlled by companies, whether they have a public source
code and accept external collaboration or not, can be run with any
software engineering method established in the company since it can be
enforced to the employees of this company. Some methods will work
better to attract external contributions but the company still
controls its own team and can maintain the software without external
collaboration.

Considering this scope, it is important to characterize the people
involved in such kind of projects. In 2002, the FLOSS Project
\cite{url:flossproject} published a report about a survey they
conducted regarding FOSS contributors. Their collected data
\cite{url:flossdata} shows that 78.77\% of the contributors are
employed or self-employed (question 42) and that only 50.82\% of the
OS community are software developers while 24.76\% do not earn their
main income with software development (question 10).  In addition to
those results, the survey presents the fact that 78.78\% of the
collaborators consider their OS tasks more joyful (question 22.2) than
their regular activities and 42.3\% also consider them better
organized (question 22.4). As an outcome of those results, we could
say that OS contributors perceive their activities both pleasurable
and effective.

% Rever essas porcentagem para trocar por algo mais interessante

Another survey \cite{reis2003} points out that 74\% of open source
projects have teams with up to 5 people and 62\% of the contributors
work with each other over the Internet and never met physically.  It
is therefore critical for those projects to have an adequate software
process that fits those characteristics and is not a burden on the
volunteer work.

% Introdução (bla bla)
% motivação (ligações de MA com OS)
% Pesquisas (objetivo e conteúdo)
% Resultados
% Propostas e conclusões

\section{How closely related are Open source and Agile?}
\label{sec:relation}

In Martin Fowler's first version of ``The New Methodology''
\cite{url:fowler2000orig}, he included open source software
development as part of the new methodology of software development
along with now well known Agile methods. He decided to remove it from
the final article because Eric Raymond's description of the
development process from ``The Cathedral and the Bazaar''
\cite{raymond1999} is not very well define and is closer to an
experience report than to a description of a process. However,
Raymond's text presents several actions that could be related to the
agile manifesto \cite{url:agilemanifesto}. 

OS communities are created are composed of individuals interacting between each other 

\section{Surveys}
\label{sec:surveys}

\subsection{To the FLOSS community}
\label{subsec:floss-survey}


\subsection{To the Agile community}
\label{subsec:agile-survey}

\section{Survey results}
\label{sec:results}

\subsection{Individual results from the FLOSS community}
\label{subsec:floss-results}

\subsection{Individual results from the Agile community}
\label{subsec:agile-results}

\subsection{Crossed results}
\label{subsec:crossed-results}

\section{Conclusion}
\label{sec:conclusion}


\subsubsection*{Acknowledgments.}

This work was supported by the QualiPSo project \cite{url:qualipso}.

\begin{thebibliography}{5}

\bibitem{report:dempsey1999} Bert J Dempsey and Debra Weiss and Paul
  Jones and Jane Greenberg: A quantitative profile of a community of
  open source Linux developers (1999)

\bibitem{url:agilemanifesto} Kent Beck and Alistair Cockburn and Ward
  Cunningham and Martin Fowler and Ken Schwaber and al.: Manifesto for
  Agile Software Development, http://agilemanifesto.org (2001)

\bibitem{conference:oopsla2007} Dennis Mancl and Steven Fraser and
  William Opdyke: No silver bullet: a retrospective on the essence and
  accidents of software engineering (2007)

\bibitem{brooks1987} Frederick P. Brooks, Jr.: No Silver Bullet:
  Essence and Accidents of Software (1987)

\bibitem{gabriel2005} Ron Goldman and Richard P. Gabriel: Innovation
  Happens Elsewhere: Open Source as Business Strategy (2005)

\bibitem{XP2002} Kent Beck and Cynthia Andres: Extreme Programming
  Explained: Embrace Change, 2nd Edition (2004)

\bibitem{schwaber2004} Ken Schwaber: Agile Project Management with
  Scrum (2004)

\bibitem{cockburn2002} Alistair Cockburn: Agile Software Development
  (2002)

\bibitem{ohno1998} Taiichi Ohno: Toyota Production System: Beyond
  Large-Scale Production (1998)

\bibitem{poppendieck2005} Mary Poppendieck and Tom Poppendieck:
  Introduction to Lean Software Development (2005)

\bibitem{url:fowler2000orig} Martin Fowler: The New Methodology,
  http://martinfowler.com/articles/newMethodologyOriginal.html

\bibitem{fogel2005} Karl Fogel: Producing Open Source Software (2005)

\bibitem{url:flossproject} International Institute of Infonomics -
  University of Maastricht: Free/Libre/Open Source Software: Survey
  and Study - Report, http://www.flossproject.org/report/

\bibitem{url:flossdata} International Institute of Infonomics -
  University of Maastricht: Free/Libre/Open Source Software: Survey
  and Study - Report, http://www.flossproject.org/floss1/stats.html

\bibitem{reis2003} Christian Robottom Reis: Caracteriza\c{c}\~{a}o de
  um Processo de Software para Projetos de Software Livre (2003)

\bibitem{raymond1999} Eric S. Raymond: The Cathedral \& the Bazaar:
  Musings on {Linux} and Open Source by an Accidental Revolutionary
  (1999)

\bibitem{oram2007} Andy Oram: Why Do People Write Free Documentation?
  Results of a Survey (2007)

\bibitem{riehle2007} Dirk Riehle: The Economic Motivation of Open
  Source Software: Stakeholder Perspectives (2007)

\bibitem{sutherland2007} Jeff Sutherland and Anton Viktorov and Jack
  Blount and Nikolai Puntikov: Distributed Scrum: Agile Project
  Management with Outsourced Development Teams (2007)

\bibitem{maurer2002} Frank Maurer: Supporting Distributed Extreme
  Programming (2002)

\bibitem{url:beck2008} Kent Beck: Tools for Agility,
  http://www.microsoft.com/downloads/details.aspx?FamilyID=ae7e07e8-0872-47c4-b1e7-2c1de7facf96
  (2008)

\bibitem{nagappan2003} Nachiappan Nagappan and Prashant Baheti and
  Laurie Williams and Edward Gehringer and David Stotts: Virtual
  Collaboration through Distributed Pair Programming (2003)

\bibitem{url:north2006} Dan North: Behaviour Driven Development,
  http://dannorth.net/introducing-bdd

\bibitem{sato2007} Danilo Sato and Alfredo Goldman and Fabio Kon:
  Tracking the Evolution of Object-Oriented Quality Metrics on Agile
  Projects (2007)

\bibitem{surowiecki2004} J. Surowiecki: The Wisdom of Crowds: Why the
  many are smarter than the few and how collective wisdom shapes
  business, economies, societies, and nations (2004)

\bibitem{tapscott2006} Don Tapscott and Anthony D. Williams:
  Wikinomics: How Mass Collaboration Changes Everything (2006)

\bibitem{benkler2006} Yochai Benkler: The Wealth of Networks: How
  Social Production Transforms Markets and Freedom (2006)

\bibitem{url:qualipso} Qualipso | Trust and Quality in Open Source
  systems, http://www.qualipso.org/

\end{thebibliography}

\end{document}
